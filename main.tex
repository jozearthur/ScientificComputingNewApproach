\documentclass[paper=9in:6in,pagesize=pdftex,headinclude=on,footinclude=on,10pt,bibtotoc,pointlessnumbers,normalheadings,DIV=9,twoside=false]{scrbook}

\areaset[0.50in]{4.5in}{8in}
\KOMAoptions{DIV=last}

\usepackage{trajan}
 
\usepackage[utf8x]{inputenc}
\usepackage[T1]{fontenc}
\usepackage{upgreek}
\usepackage{float}
\usepackage[normal,font={footnotesize,it}]{caption}

\usepackage[sc]{mathpazo}
\linespread{1.05} 
\usepackage[english]{babel}
\usepackage{amsthm}
\usepackage{amsmath}
\usepackage{MnSymbol}
\usepackage{wasysym}
\usepackage{amsfonts}
\newtheorem{theorem}{Theorem}
\renewcommand\qedsymbol{$\blacksquare$}


\begin{document}
\date{}

\begin{center}
\begin{large}
 \textbf{Perceptron Convergence Theorem\\}
\end{large}
\end{center}
\begin{text} 
We will prove the Rosenblatt theorem for convergence
of a learning algorithm. If there is a solution, this is,
if there is a vector \textbf{B} that separates a set of vectors in the correct categories with a certain margin k, then in a finite number of steps it is possible to find some vector that also achieves this separation. We call hit when $\textbf{J}(t)\cdot \boldsymbol{\zeta_{\mu}}\sigma_{\mu} > \sqrt{N}k$, otherwise we call a miss. \\

The vectors of the training set will be considered sequentially and presented to the perceptron, if the classification is correct nothing is done. If an error occurs, a change in the weight vector
$\mathbf{J} (t)$ is required. The process is repeated with the next element of list until convergence is achieved. We start from $ \mathbf{J}(0) = 0$. The learning dynamics is given by:\\
\end{text}

\begin{center}
    $\mathbf{J}(t+1) = \mathbf{J}(t) + \frac{f_\mu}{\sqrt{N}} \cdot  \boldsymbol{\zeta_{\mu}}\sigma_{\mu}$
\end{center}

\begin{text}
where $f_\mu$ is 0 if we have a hit and 1 otherwise. This is an error correction algorithm. \\

We suppose that exists $\mathbf{B}$ where $\mathbf{B} \cdot \mathbf{B}=N$ such that: \\


\end{text}


\begin{text}
  \ \hspace{1.5in} $\frac{1}{\sqrt{N}}\boldsymbol{\zeta_{\mu}}\cdot \mathbf{B}$ $\sigma_{\mu} \geq k > 0 $  \hspace{3cm} (1.1) \\
\end{text}

\begin{text}
At a given time $t$, we have $F = \sum f_{\mu}$ and $\mathbf{J} (t) =  \sum \frac{1}{\sqrt{N}} f_{\mu} \boldsymbol{\zeta_{\mu}}\sigma_{\mu}$, respectively the number of effective learning steps and the perceptron weights. We want to show that $F$ it remains finite. Multiplying \textbf{(1.1)} by $f_{\mu} $ and summing:  
\end{text}

\begin{center}
    $\sum \frac{1}{\sqrt{N}} f_{\mu} \boldsymbol{\zeta_{\mu}}\cdot \mathbf{B}$ $\sigma_{\mu} = \mathbf{J}\cdot \mathbf{B} \geq Fk  $
\end{center}

\begin{text}
We introduce $\rho = \frac{\mathbf{J}\cdot \mathbf{B}}{\mid \mathbf{J}\mid \cdot \mid \mathbf{B} \mid }$
. It is easy to show that $\rho$ is between −1 and 1, because
is the cosine of the angle between the two vectors. $\mathbf{J}\cdot \mathbf{B} = \rho \sqrt{N} \mid \mathbf{J} \mid $ (because $\mathbf{B} \cdot \mathbf{B}=N$). Squaring the inequality we have:
\end{text}

\begin{center}
    $\rho^2 N \mid \mathbf{J} \mid^2 \geq (Fk)^2$\\
\end{center}

\begin{text}
Analysing the learning dynamic we have: \\
\end{text}

\begin{center}
    $\mid \mathbf{J}(t+1) \mid ^2 = \mid \mathbf{J}(t) \mid^2 + 2 \frac{1}{\sqrt{N}} \mathbf{J}(t)\cdot \boldsymbol{\zeta_{\mu}} \sigma_{\mu} + \frac{1}{N} \mid \boldsymbol{\zeta_{\mu}} \mid^2$
\end{center}

\begin{text}
We can use the scale $\mid \boldsymbol{\zeta_{\mu}} \mid^2 = N$ because the classification does not depend of modules. \\
\end{text}

\begin{center}
    $\mid \mathbf{J}(t+1) \mid ^2 = \mid \mathbf{J}(t) \mid^2 + 2 \frac{1}{\sqrt{N}} \mathbf{J}(t)\cdot \boldsymbol{\zeta_{\mu}} \sigma_{\mu} + 1 $
\end{center}

\begin{text}
Analysing the element $\mu$, if he is changing the dynamic of \textbf{J} then he was not correct classified. It was a miss, so we have $ \frac{1}{\sqrt{N}}\textbf{J}(t)\cdot \boldsymbol{\zeta_{\mu}}\sigma_{\mu} < k$ hence:
\end{text}

\begin{center}
    $\mid \mathbf{J}(t+1) \mid ^2 \leq \mid \mathbf{J}(t) \mid + 2k +1 \leq F(2k+1)$
\end{center}

\begin{text}
Because every increase step is smaller than 2k+1, so if were given F steps, than $F(2k+1) \geq \mid \textbf{J}(t) \mid $ . Finally we have:
\end{text}

\begin{center}
    $(Fk)^2 \leq N \mid \mathbf{J}(t+1) \mid ^2 \leq NF(2k+1)$ \\
    \ \\
    $F \leq N(2k^{-1} + k^{-2})$
\end{center}

\begin{text}
So F is limited. \\
\qed
\end{text}





\end{document}
