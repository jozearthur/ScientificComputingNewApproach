\documentclass[paper=9in:6in,pagesize=pdftex,headinclude=on,footinclude=on,10pt,bibtotoc,pointlessnumbers,normalheadings,DIV=9,twoside=false]{scrbook}

\areaset[0.50in]{4.5in}{8in}
\KOMAoptions{DIV=last}

\usepackage{trajan}
 
\usepackage[utf8x]{inputenc}
\usepackage[T1]{fontenc}
\usepackage{upgreek}
\usepackage{float}
\usepackage[normal,font={footnotesize,it}]{caption}

\usepackage[sc]{mathpazo}
\linespread{1.05} 


\usepackage[english]{babel}
\usepackage{amsthm}
\usepackage{amsmath}
\usepackage{MnSymbol}
\usepackage{wasysym}
\usepackage{amsfonts}
\newtheorem{theorem}{Theorem}
\renewcommand\qedsymbol{$\blacksquare$}


\begin{document}
\date{}


\begin{large} 
 \textbf{Exercise}
\end{large} 
\begin{itemize} 
\item In a dark chamber 20 coins (10 heads and 10 tails) rests on the ground. We want to separate 2 sets of 10 coins with the same number of tails. We can not see the coins, we can only flip each one. How would you separate the coins?
\end{itemize}

\begin{large}
\textit{Solution:\\}
\end{large}

\begin{text}
We first pick up 10 random coins, in this set we have x heads and $10-x$ tails and in the other set we have y heads and $10-y$ tails. 
\end{text}

\begin{text}
Since $x+y=10$ (there was 10 heads in the initial set) the first set is compound by $x$ heads and $y$ tails and the second one is compound by $y$ heads and $x$ tails. If we flip all the 10 coins of one set, we guarantee that both sets have the same number of tails and heads.

\end{text}







\end{document}
