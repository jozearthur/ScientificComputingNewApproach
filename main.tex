\documentclass[paper=9in:6in,pagesize=pdftex,headinclude=on,footinclude=on,10pt,bibtotoc,pointlessnumbers,normalheadings,DIV=9,twoside=false]{scrbook}

\areaset[0.50in]{4.5in}{8in}
\KOMAoptions{DIV=last}

\usepackage{trajan}
 
\usepackage[utf8x]{inputenc}
\usepackage[T1]{fontenc}
\usepackage{upgreek}
\usepackage{float}
\usepackage[normal,font={footnotesize,it}]{caption}

\usepackage[sc]{mathpazo}
\linespread{1.05} 


\usepackage[english]{babel}
\usepackage{amsthm}
\usepackage{amsmath}
\usepackage{MnSymbol}
\usepackage{wasysym}
\usepackage{amsfonts}
\newtheorem{theorem}{Theorem}
\renewcommand\qedsymbol{$\blacksquare$}


\begin{document}
\date{}


\begin{large} 
 \textbf{Solving Gaussian Integrals \\}
\end{large} 

\begin{text}
In this section we will show how to solve some Gaussian integrals, first of all, lets look to the simple one:  $I = \int\limits_{-\infty}^{\infty} e^{-x^2} dx$. \\
\end{text}

\begin{center}
    $I^2 = \int\limits_{-\infty}^{\infty} e^{-x^2} dx \cdot \int\limits_{-\infty}^{\infty} e^{-x^2} dx = \int\limits_{-\infty}^{\infty} e^{-x^2} dx \cdot \int\limits_{-\infty}^{\infty} e^{-y^2} dy$
\end{center}

\begin{text}
We now introduce an other 2 variables $r^2=x^2+y^2$ and $\theta = \arctan \frac{x}{y}$ hence:
\end{text}

\begin{center}
    $I^2 = \int\limits_{-\infty}^{\infty} \int\limits_{-\infty}^{\infty} e^{-(x^2+y^2)} dxdy = \int\limits_{0}^{\infty}\int\limits_{0}^{2\pi} r e^{-r^2}drd\theta = \int\limits_{0}^{2\pi} d\theta \int\limits_{0}^{\infty} re^{-r^2} dr$
\end{center}


\begin{text}
Let be $u=-r^2 \implies 2rdr = -du$, so we have:
\end{text}

\begin{center}
    $I^2 = -\pi \int\limits_{0}^{-\infty} e^u du = \pi$ \\
    
    \ \\
    
   $ \int\limits_{-\infty}^{\infty} e^{-x^2} dx = \sqrt{\pi}$ \\
\end{center}

\begin{text}

To find $\int\limits_{-\infty}^{\infty} e^{\frac{-(x-\mu)^2}{2\sigma^2}}dx$, we make a simple substitution $u = \frac{(x-\mu)}{\sqrt{2}\sigma}$: \\

\end{text}

\begin{center}
    $\int\limits_{-\infty}^{\infty} e^{\frac{-(x-\mu)^2}{2\sigma^2}}dx = \int\limits_{-\infty}^{\infty} \sqrt{2}\sigma e^{-u^2}du = \frac{}{} = \sqrt{2\pi}\sigma$. \\
\end{center}

\begin{text}
To find the expected value of x we need to solve the integral $\int\limits_{-\infty}^{\infty} \frac{1}{\sqrt{2\pi}\sigma} x e^{\frac{-(x-\mu)^2}{2\sigma^2}}dx $ and again we will use a simple substitution $\frac{(x-\mu)^2}{2\sigma^2} = u^2 \implies udu = \frac{1}{\sqrt{2\pi}\sigma} (xdx -  \mu dx)$, so we have: \\
\end{text}

\begin{center}
$\int\limits_{-\infty}^{\infty} \frac{1}{\sqrt{2\pi}\sigma} x e^{\frac{-(x-\mu)^2}{2\sigma^2}}dx = \int\limits_{-\infty}^{\infty} \frac{1}{\sqrt{2\pi}\sigma}  e^{\frac{-(x-\mu)^2}{2\sigma^2}} \mu dx + \int\limits_{-\infty}^{\infty} u  e^{-u^2}du  $    
\end{center}

\begin{text}
We already solve the first integral, it is equal $\mu$, the second one it is just 0 because is an integral of odd function. Finally we have:
\end{text}

\begin{center}
    $<x> = \int\limits_{-\infty}^{\infty} \frac{1}{\sqrt{2\pi}\sigma} x e^{\frac{-(x-\mu)^2}{2\sigma^2}}dx = \mu $
\end{center}






\end{document}
