\documentclass[paper=9in:6in,pagesize=pdftex,headinclude=on,footinclude=on,10pt,bibtotoc,pointlessnumbers,normalheadings,DIV=9,twoside=false]{scrbook}

\areaset[0.50in]{4.5in}{8in}
\KOMAoptions{DIV=last}

\usepackage{trajan}
 
\usepackage[utf8x]{inputenc}
\usepackage[T1]{fontenc}
\usepackage{upgreek}
\usepackage{float}
\usepackage[normal,font={footnotesize,it}]{caption}

\usepackage[sc]{mathpazo}
\linespread{1.05} 


\usepackage[english]{babel}
\usepackage{amsthm}
\usepackage{amsmath}
\usepackage{MnSymbol}
\usepackage{wasysym}
\usepackage{amsfonts}
\newtheorem{theorem}{Theorem}
\renewcommand\qedsymbol{$\blacksquare$}




\begin{document}
\date{}




\begin{theorem} 
  let $A \in \Sigma$ so $P(A)+P(A^c)=1 $, $ \forall $ A $\in \Sigma$  where $A^c$ is the logical negation of A\\
  \end{theorem}
  
  \begin{proof}  Since $\omega$ is our sample space $A^c$ is defined as $\omega/A$ we have: 
  
  \begin{center}
  $P(A)+P(A^c)=P(A)+P(\omega/A)=P(\omega)$
  \end{center}
  
  By the definition of probability $P(\omega)=1$, so we have $P(A)+P(A^c)=1  $\\
  \end{proof}
  
  \begin{theorem} The probability of the empty set is 0, $P(\emptyset)=0$\\
  \end{theorem}
  
  \begin{proof} Let $A \in \Sigma$, since $A\cup\emptyset=A$ we have:
  
  \begin{center} $P(A\cup\emptyset)=P(A)+P(\emptyset)=P(A)$\\ $P(\emptyset)=P(A)-P(A)=0$
  \end{center}
  \end{proof}
  
  \begin{theorem} Let $A_1, A_2 \dots{} A_N \in \Sigma$, if $A_i$ $\cap$ $A_j=\emptyset$ $\forall i,j$  1 $\leq$ i,j $ \leq$ N so we have $P(\bigcup\limits_{i = 1 }^N A_i) = 	\sum_{i=1}^{N} P(A_i)$\\
  \end{theorem}
  
  \begin{proof}The equation is true for N=2, since this reduce to the case in the definition of probability. \\
  
 Assume that $P(\bigcup\limits_{i = 1 }^N A_i) = 	\sum_{i=1}^{N} P(A_i)$ is true for $N$. Since $A_{N+1} \cap \bigcup\limits_{i = 1 }^N A_i=\emptyset$ we have:
 
 \begin{center} $P(\bigcup\limits_{i = 1 }^N A_i)+P(A_{N+1})=\sum_{i=1}^{N} P(A_i)  +P(A_{N+1}) =\sum_{i=1}^{N+1} P(A_i)$ \\ $P(\bigcup\limits_{i = 1 }^N A_i)+P(A_{N+1})= P([\bigcup\limits_{i = 1 }^N A_i] \cup A_{N+1}) = P(\bigcup\limits_{i = 1 }^{N+1} A_i)$ hence:\\ $P(\bigcup\limits_{i = 1 }^{N+1} A_i)=\sum_{i=1}^{N+1} P(A_i)$ $\forall N \in  \mathbb{N}$

 
 \end{center}




\end{proof}







\end{document}
